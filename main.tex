%% 
%% Copyright 2019-2021 Elsevier Ltd
%% 
%% This file is part of the 'CAS Bundle'.
%% --------------------------------------
%% 
%% It may be distributed under the conditions of the LaTeX Project Public
%% License, either version 1.2 of this license or (at your option) any
%% later version.  The latest version of this license is in
%%    http://www.latex-project.org/lppl.txt
%% and version 1.2 or later is part of all distributions of LaTeX
%% version 1999/12/01 or later.
%% 
%% The list of all files belonging to the 'CAS Bundle' is
%% given in the file `manifest.txt'.
%% 
%% Template article for cas-sc documentclass for 
%% single column output.

\documentclass[a4paper,fleqn]{cas-sc}

% If the frontmatter runs over more than one page
% use the longmktitle option.
%\documentclass[a4paper,fleqn,longmktitle]{cas-sc}

%\usepackage[numbers]{natbib}
\usepackage[authoryear]{natbib}
%\usepackage[authoryear,longnamesfirst]{natbib}



%%=====
%%=====  Packages included by us
%%=====
% Took this part from elsarticle example
\usepackage[mathlines]{lineno} 
\modulolinenumbers[1]

\usepackage{anyfontsize}
\usepackage{lipsum}
\usepackage{xcolor} 
\usepackage{csquotes} % In order to use \enquote
\usepackage{float}
\usepackage{textcomp}
\usepackage{relsize} % In order to use \mathsmaller
\usepackage{upgreek} % Need extra font for greek letters
\usepackage[section]{placeins} % Keep figures within same section
\usepackage{tabularx} % Table with automatic line break within a cell
\usepackage{caption}
\usepackage{hyperref} 
\usepackage{afterpage}

% I wanted amsfonts for mathbb (i.e. with serif), not the one included in cas-sc
\DeclareSymbolFontAlphabet{\mathbb}{AMSb}

% Same thing for mathcal
\DeclareSymbolFont{cmsymbols}{OMS}{cmsy}{m}{n}
\SetSymbolFont{cmsymbols}{bold}{OMS}{cmsy}{b}{n}
\DeclareSymbolFontAlphabet{\mathcal}{cmsymbols}

% Hyperbolyc secant was missing
\DeclareMathOperator{\sech}{sech}

% Override \vec with an invocation of \vect.
\newcommand{\vect}[1]{\boldsymbol{\mathbf{#1}}}

\let\stdvec\vec
\renewcommand{\vec}[1]{\vect{#1}}

% ---- Change the real and imaginary symbols to something that looks nicer (in our opinion)
\renewcommand{\Re}{\operatorname{Re}}
\renewcommand{\Im}{\operatorname{Im}}

% In order to make smaller subscripts. For instance, $\varphi_O$ was too big
\newcommand{\sbs}[1]{_{\!\mathsmaller{#1}}}

% Useful for some tables
\newcolumntype{Y}{>{\centering\arraybackslash}X}


%%%% Solution found at
%%%% https://tex.stackexchange.com/questions/436011/linenomath-printing-extra-numbers-on-last-line-of-multline-align-flalign-envir
%%%% for the problem of numbering nested equations
\usepackage{etoolbox} %% <- for \cspreto, \csappto, \patchcmd, \pretocmd, \apptocmd

%% Patch 'normal' math environments:
\newcommand*\linenomathpatch[1]{%
  \cspreto{#1}{\linenomath}%
  \cspreto{#1*}{\linenomath}%
  \csappto{end#1}{\endlinenomath}%
  \csappto{end#1*}{\endlinenomath}%
}

%% Patch AMS math environments:
\newcommand*\linenomathpatchAMS[1]{%
  \cspreto{#1}{\linenomathAMS}%
  \cspreto{#1*}{\linenomathAMS}%
  \csappto{end#1}{\endlinenomath}%
  \csappto{end#1*}{\endlinenomath}%
}

%% Definition of \linenomathAMS depends on whether the mathlines option is provided
\expandafter\ifx\linenomath\linenomathWithnumbers
  \let\linenomathAMS\linenomathWithnumbers
  %% The following line gets rid of an extra line numbers at the bottom:
  \patchcmd\linenomathAMS{\advance\postdisplaypenalty\linenopenalty}{}{}{}
\else
  \let\linenomathAMS\linenomathNonumbers
\fi

\linenomathpatch{equation}
\linenomathpatchAMS{gather}
\linenomathpatchAMS{multline}
\linenomathpatchAMS{align}
\linenomathpatchAMS{alignat}
\linenomathpatchAMS{flalign}


% This essentially duplicates \substack, but adding an alignment point.
\makeatletter
\newcommand{\subalign}[1]{%
  \vcenter{%
    \Let@ \restore@math@cr \default@tag
    \baselineskip\fontdimen10 \scriptfont\tw@
    \advance\baselineskip\fontdimen12 \scriptfont\tw@
    \lineskip\thr@@\fontdimen8 \scriptfont\thr@@
    \lineskiplimit\lineskip
    \ialign{\hfil$\m@th\scriptstyle##$&$\m@th\scriptstyle{}##$\hfil\crcr
      #1\crcr
    }%
  }%
}
\makeatother


% Keep floats inside subsection
\let\Oldsection\section
\renewcommand{\section}{\FloatBarrier\Oldsection}

\let\Oldsubsection\subsection
\renewcommand{\subsection}{\FloatBarrier\Oldsubsection}

%%=====
%====== END OF MY MODIFICATIONS
%%=====



\begin{document}
\let\WriteBookmarks\relax
\def\floatpagepagefraction{1}
\def\textpagefraction{.001}

% Short title
\shorttitle{Discussion on some relevant aspects for the experimental and numerical analysis of FWTs}

% Short author
\shortauthors{L. H. S. do Carmo, P. C. de Mello, R. M. Monaro and A. N. Simos}

% Main title of the paper
\title [mode = title]{A discussion on some relevant aspects for the experimental and numerical analysis of floating wind turbines based on a test case}

% Title footnote mark
% eg: \tnotemark[1]
%\tnotemark[<tnote number>] 

% Title footnote 1.
% eg: \tnotetext[1]{Title footnote text}
%\tnotetext[<tnote number>]{<tnote text>} 

% First author
%
% Options: Use if required
% eg: \author[1,3]{Author Name}[type=editor,
%       style=chinese,
%       auid=000,
%       bioid=1,
%       prefix=Sir,
%       orcid=0000-0000-0000-0000,
%       facebook=<facebook id>,
%       twitter=<twitter id>,
%       linkedin=<linkedin id>,
%       gplus=<gplus id>]
\author[1]{Lucas H. S. do Carmo}[orcid=0000-0001-8744-1391]

% Corresponding author indication
\cormark[1]
\cortext[1]{Corresponding author.}

% Footnote of the first author
%\fnmark[1]

% Email id of the first author
\ead{lucas.carmo@usp.br}

% URL of the first author
%\ead[url]{lucas.carmo@usp.br}

% Credit authorship
\credit{Conceptualization, Methodology, Software, Validation, Formal analysis, Writing -- original draft}

% Address/affiliation
\affiliation[1]{organization={University of São Paulo},
            addressline={Av. Prof. Mello Moraes, 2231}, 
            city={São Paulo},
            %citysep={}, % Uncomment if no comma needed between city and postcode
            postcode={05508-030}, 
            %state={SP},
            country={Brazil}}

\author[1]{Pedro C. de Mello}[orcid=0000-0003-2621-9644]
\credit{Conceptualization, Experiments}
\author[1]{Renato M. Monaro}[orcid=0000-0002-7453-8650]
\credit{Conceptualization, Experiments}
\author[1]{Alexandre N. Simos}[orcid=0000-0002-1879-5468]
\credit{Conceptualization, Formal analysis, Writing -- review, Supervision}



% For a title note without a number/mark
%\nonumnote{}

% Here goes the abstract
\begin{abstract}
A \lipsum[1-1]
\end{abstract}

% Keywords
% Each keyword is seperated by \sep
\begin{keywords}
    \sep Floating wind turbines \sep Model tests \sep Numerical modeling \sep Software-in-the-loop
\end{keywords}
  
\maketitle

% Main text
\linenumbers
\section{Introduction} \label{sec:introduction}
Floating offshore wind turbines (FOWTs) have been the subject of numerous studies due to the possibility of exploiting the vast wind resources located in deep waters. As an emerging technology, the growth of the wind energy industry depends on FOWTs achieving more competitive costs, which has pushed for larger rotors and new designs for both floaters and moorings.

Since FOWTs are complex structures, their design requires the evaluation of performance and structural integrity for a myriad of environmental conditions (wind, wave, current, among others) and operating conditions (power production, normal shut down, fault conditions, etc.). Due to their intricate dynamics, this procedure requires modelling software capable of accounting for the couplings between aerodynamics, hydrodynamics, controls, moorings and structural behavior, which are commonly referred as aero-hydro-servo-elastic tools. A substantial effort has been made to validate these software, as exemplified by the OC3~\citep{jonkman2010report}, OC4~\citep{OC42014} and OC5~\citep{OC52017} projects, but this is still an ongoing development.

In fact, the experiments required to validate the numerical tools, usually performed in model scale, are far from an easy task, for it is impossible to keep all the dimensionless parameters that describe the different physical aspects of the problem. For instance, while the scaling of the waves requires that the Froude number ($\textrm{Fr} = U^2/(gL)$, with $U$ a characteristic speed, $L$ a characteristic length and $g$ the gravitational acceleration) be conserved, the aerodynamic loads are governed by the Reynolds number ($\textrm{Re} = UL/\nu$, with $\nu$ the kinematic viscosity). To work around this incompatibility, some alternatives have been tried to perform tests with both wind and waves, and a thorough review of experimental techniques for floating offshore wind turbines can be found in \textcolor{red}{otter}. For instance, some works have used a Froude scaled rotor with higher wind speeds, with the wind generated by fans, in order to obtain the correct rotor thrust \textcolor{red}{111, 115, 116}, but this approach has the downside that either the tip speed ratio (TSR) or the excitation frequencies are not preserved. Others have employed performance scaled rotors \textcolor{red}{3, 120, 121, 122}, in the sense that the rotors were redesigned with geometrically modified airfoils to compensate for the low Reynolds number obtained in a Froude scale experiment. 

A different line of thought 



O SIL com atuador é muito simples em questão de equipamento, já que não precisa de ventiladores e nem das winches


- ROTORES EM ESCALA, ETC. 
--- Roddier et al, 2010 -> Drag disc
--- Martin et al, 2014 
    -> Froude scaled rotor with higher wind speeds, but wrong thrust
    -> Increase rotor velocity to keep TSR, changes excitation frequencies
    -> Geommetrically modified airfoils to compensate for low Reynolds Number -> Geommetrically scaling of the rotor

- SIL C/ vento

- E OS COM CABOS


Vittori et al, 2022: SIL c/ thruster
Thys et al, 2021: Cabos



Explicar nosso objetivo, que é duplo: verificar nossos modelos numéricos e, concomitantemente, desenvolver a capacidade do tanque de provas numérico em realizar esse tipo de ensaio.

Ensaio com SIL com algumas limitações

Na primeira parte, a ideia é verificar aspectos importantes da hidrodinâmica (que é a parte que é fisicamente capturada no ensaio) que o modelo numérico tem que levar em conta. Mais especificamente, é avaliada a importância de elementos de Morison retangulares p/ modelagem do pontoon; a importância de forças de segunda-ordem tanto na horizontal quanto vertical (conforme já sabido na literatura); e mostra-se que levar em conta a inclinação média do casco devido ao vento não é mto importante.

Na segunda parte, o objetivo é avaliar o quão importante são aspectos que foram deixados de lado na modelagem aerodinâmica do ensaio, o que é feito numericamente comparando o modelo que é fiel às condições de ensaio (apenas thrust e pás consideradas rígidas) com um modelo numérico em que as forças aerodinâmicas são calculadas nos seis graus de liberdade e a flexibilidade das pás é considerada (embora de forma simples com o elastodyn. Preciso estudar em que situações seria necessário usar o beamdyn).
\section{Description of the prototype and the experimental setup} \label{sec:description_experiment}
The FOWTC hull concept is the result of the parametrical optimization procedure reported by \citet{mas2022parametric}. Following the two objectives given in Section~\ref{sec:introduction}, the experiments were conducted in such a way that they were as close as possible to the design conditions in order to allow for the verification of the numerical methods. Two main differences are, however, present: the water depth, which had to be reduced from $600\,\text{m}$ to $233\,\text{m}$, and the limitations of the software-in-the-loop approach, outlined in Section~\ref{sec:introduction} and discussed in details in Section~\ref{sec:impact_simplifications}.

The experimental campaign was conducted at the wave basin of the Numerical Offshore Tank of the University of São Paulo (TPN-USP), a squared $14\,\text{m}\times 14\,\text{m} \times 4\,\text{m}$ (length, width, depth) tank equipped with 152 active-absorption flap-type wave generators that is shown in Figure~\ref{fig:description_experiment:tanque}.
\begin{figure}[!hbtp]
	\centering
	\includegraphics[width=0.5\columnwidth]{./figures/CH-tpn.jpg}%
	\caption{Wave basin of the Numerical Offshore Tank of the University of São Paulo.} \label{fig:description_experiment:tanque}%
\end{figure}%


\subsection{Main properties of the FOWTC}
%- Caracteristicas da FOWT, RNA, ancoragem
The FOWTC concept consists of a semi-submersible hull with a $14.1\,\text{m}$ diameter central column attached to three $17.0\,\text{m}$ diameter columns arranged as an equilateral triangle, connected by rectangular pontoons $17.0\,\text{m}$ high and $6.0\,\text{m}$ wide, which was built in a 1:70 scale for the experiments. 

During design, the RNA of the IEA 15MW turbine \citep{gaertner2020definition} and the tower of the UMaine VolturnUS-S floater~\citep{allen2020definition} were mounted on top of the central column, but their inertial and structural characteristics were not preserved during the tests. Instead, the global inertial properties of the whole FOWT were matched by using a ballast system that included a moving set of weights that could be adjusted along a fuse located on top of the tower. A picture of the model is given in Figure~\ref{fig:description_experiment:modelo}, while its main properties are listed in Table~\ref{tab:description_experiment:FOWTC_properties}.
\begin{figure}[!hbtp]
	\centering
	\includegraphics[height=10cm]{./figures/foto_modelo.png}%
	\caption{Picture of the FOWTC model.} \label{fig:description_experiment:modelo}%
\end{figure}%
%\begin{figure}[!hbtp]
%	\centering
%	\fbox{\includegraphics[trim={11cm 5cm 11cm 5cm},clip, height=10cm]{./figures/rhino_lastros.pdf}}%
%	\caption{Picture of the FOWTC model.} \label{fig:description_experiment:rhino_lastros}%
%\end{figure}%


\begin{table}[!hbtp]
	\centering
	\caption{Main properties of the FOWTC.} \label{tab:description_experiment:FOWTC_properties}   
	\begin{tabular}{lrr}
		\toprule
		& Full scale & Model Scale (1:80) \\
		\midrule
		Mass & $6 \mkern2mu 936.0 \, \text{t}$ & $13.55 \, \text{kg}$\\
		%
		Displacement & $7 \mkern2mu 351.3 \, \text{m}^3$ & $14.36 \, \text{L}$\\
		%
		Diam. of central column & $ 15.0 \, \text{m}$ & $188\,\text{mm}$ \\
		%
		Diam. of side columns & $ 9.0 \, \text{m}$ & $113\,\text{mm}$ \\			
		%
		Pitch/roll gyradius & $21.9 \, \text{m}$ & $274\,\text{mm}$ \\
		%
		Yaw gyradius & $20.3 \, \text{m}$ & $254\,\text{mm}$\\
		%
		Draft & $20.0 \, \text{m}$ & $250\,\text{mm}$ \\
		%
		KG & $15.6 \, \text{m}$ & $195 \,\text{mm}$ \\
		%
		KB & $10.0 \, \text{m}$ & $125 \,\text{mm}$  \\
		%
		BM & $8.9 \, \text{m}$ & $111 \,\text{mm}$  \\
		%
		GM & $3.3 \, \text{m}$ & $41 \,\text{mm}$  \\
        \bottomrule
        & & \\[-2pt]
		%        
        \multicolumn{3}{l}{Natural periods} \\ 
		\midrule        
		Surge/Sway & $86.3 \,\text{s}$ & $9.65 \,\text{s}$ \\
		Heave & $9.8 \,\text{s}$ & $1.09 \,\text{s}$ \\
		Pitch/Roll & $21.0 \,\text{s}$  & $2.35 \,\text{s}$\\
		Yaw & $47.0 \,\text{s}$ & $5.25 \,\text{s}$\\
		\bottomrule
	\end{tabular}%	
\end{table}%




\subsection{Software-in-the-loop approach for aerodynamic loads}
Tem que incluir o controle. Adicionar alguns resultados de teste de bancada

\subsection{Limitations of the experiment}

\subsection{Environmental conditions} \label{sec:description_experiment:envir}
- Condições de onda e vento
\section{Numerical models} \label{sec:numerical_models}
Modelos numéricos que foram feitos com diferentes graus de proximidade pro ensaio.

\section{Reproducing the experiments with numerical models} \label{sec:exp_vs_num}
As outlined in Section~\ref{sec:introduction}, the purpose of this section is to discuss the relevance of three specific points concerning the hydrodynamic modeling in the numerical simulations, in light of the experimental results: the drag forces on the pontoons (Section~\ref{subsec:exp_vs_num:drag}); the need for second-order wave forces not only for the horizontal motions, but also for the vertical dofs (Section~\ref{subsec:exp_vs_num:2nd_order}); and whether the mean hull inclination induced by the wind needs to be considered in the solution of the radiation/diffraction problem (Section~\ref{subsec:exp_vs_num:impact_inclination}). After these aspects have been discussed, Section~\ref{subsec:exp_vs_num:main_results} summarizes the main results to show the adherence between the numerical simulations and the experiment.


\subsection{The need for drag forces on the pontoons} \label{subsec:exp_vs_num:drag}
The first point concerns the drag forces on the pontoons of the hull, which are relevant not only due to the damping that they introduce, but also because drag is and important forcing term for this type of floater. These forces are modeled in OpenFAST using Morison elements, but since the current version of the software includes only circular elements, the source code was modified to account for elements with rectangular cross section. In this modified version, the drag forces per unit length are given by:
\begin{equation}
\begin{split}
	f_{Dx} &= \frac{1}{2} \rho \mkern2mu C_{Dx} \mkern2mu D_x \,||\vec{u}|| \, u_x \\[6pt]
	f_{Dy} &= \frac{1}{2} \rho \mkern2mu C_{Dy} \mkern2mu D_y \,||\vec{u}|| \, u_y
\end{split}		
\end{equation}
%
where $\vec{u}=u_x\vec{e}_1+u_y\vec{e}_2$ is the relative fluid velocity (i.e. the difference between fluid velocity and local body velocity); $\vec{e}_1$ and $\vec{e}_2$ are the unit vectors directed along the local $x$ and $y$ axes, which are normal to the lateral faces of the rectangular cylinder; $C_{Dx}$ and $C_{Dy}$ are the drag coefficients in each direction; and $D_x$ and $D_y$ are the sides of the rectangle.

The drag coefficients in each direction are obtained by matching numerical decays in surge and heave with their experimental counterparts, as illustrated in Figures~\ref{fig:exp_vs_num:drag:surge_decay}~and~\ref{fig:exp_vs_num:drag:heave_decay}. These decays were performed with the complete setup of the experiment, hence with both the mooring lines and the umbilical cables needed by the SIL method, but without either wind or waves. In the right side of each graph is presented a PQ-analysis~\citep{burmester2020}, in which the slope of the curve corresponds to the quadratic damping, while the intersection with the $y$ axis is proportional to the linear damping. It turns out that the latter, which is mostly due to wave radiation, is negligible, being relevant only when the amplitude of motion is very small (note that the experiment provides a negative value for the linear heave damping, which is clearly not physical and a consequence of the small values involved). On the other hand, the quadratic damping, which is obtained considering the drag coefficients summarized in Table~\ref{tab:exp_vs_num:drag:drag_coeffs}, is within the range of values measured in the experiments (3 repetitions in surge and 7 in heave), as shown in Table~\ref{tab:exp_vs_num:drag:quad_damp_decay} (a complete set of decay results is given in Table~\textcolor{red}{5}). Note that it would be impossible to match the damping levels in both dofs simultaneously if the pontoons were modeled using circular Morison elements, since both $D$ and $C_D$ are different for each direction. 
\begin{figure}[!hbtp]
	\centering
	\includegraphics[width=0.9\textwidth]{./figures/surge_decay_drag_pontoon.png}	
	\caption{Comparison of numerical and experimental surge decays.} \label{fig:exp_vs_num:drag:surge_decay}
\end{figure}

\begin{figure}[!hbtp]
	\centering
	\includegraphics[width=0.9\textwidth]{./figures/heave_decay_drag_pontoon.png}	
	\caption{Comparison of numerical and experimental heave decays.} \label{fig:exp_vs_num:drag:heave_decay}
\end{figure}

Though the drag coefficients obtained from decay tests are not necessarily the same required by simulations considering waves, the values provided in Table~\ref{tab:exp_vs_num:drag:drag_coeffs} have led to good results for all the wave and wind conditions analyzed in this work, as will be shown in Section~\ref{subsec:exp_vs_num:main_results}. For now, a single test case is enough to illustrate the influence of the drag forces. For that, Figures~\ref{fig:exp_vs_num:drag:ptfmsurge}~and~\ref{fig:exp_vs_num:drag:ptfmheave} present the surge and heave motions of the FOWT under the action of the IRR-I1 wave ($T_P=14.5\,\text{m}$ and $H_S=9.3\,\text{m}$) and an incidence of 60\textdegree{}, without wind effects. To evidence the importance of the drag forces, the plots include results of simulations performed with circular Morison elements calibrated for the opposite dof, i.e. the results labeled \enquote{OpenFAST - Circ. pontoon S} presented in the heave graphs correspond to a circular pontoon with $D=6.0\,\text{m}$ and $C_D=2.50$, which are the values obtained from the surge decays; conversely, \enquote{OpenFAST - Circ. pontoon H} correspond to a circular pontoon calibrated using the heave decays.

\begin{table}[!hbtp]
	\caption{Resulting drag coefficient for each Morison element and its main dimensions.}\label{tab:exp_vs_num:drag:drag_coeffs}
	\begin{tabular}{lrrr}
		\toprule
		& Length & Diameter & $C_D$ \\
		\midrule
		Central column & 18.6 & 14.1 & 0.61 \\
		Offset columns & 18.6 & 17.0 & 0.61 \\
		Pontoons - Vertical & 23.0 & 17.0 & 12.00 \\
		Pontoons - Horizontal & 23.0 & 6.0 & 2.50 \\
		\bottomrule
\end{tabular}
\end{table}

\begin{table}[!hbtp]
	\caption{Comparison between the experimental and numerical values of the nondimensional quadratic drag coefficient ($B_Q/M$) obtained in the surge/heave decays.}\label{tab:exp_vs_num:drag:quad_damp_decay}
	\begin{tabular}{crr}
		\toprule
		& Experiment & OpenFAST \\
		\midrule
		Surge & $0.021\pm0.002$ & 0.019 \\
		Heave & $0.145\pm0.022$ & 0.127 \\
		\bottomrule
	\end{tabular}
\end{table}

The first set of graphs, Figure~\ref{fig:exp_vs_num:drag:ptfmsurge}, shows that the difference in drag models has a significant impact on the surge motion only around its resonance frequency, which is expected due to the different damping levels. However, the slow surge motion is underpredicted even by the OpenFAST model with rectangular pontoons and drag coefficients calibrated from decay tests. The main hypothesis is that this is related to an underprediction of the low-frequency force rather than an overprediction of damping, following the findings by \citet{wang2022oc6}, who showed that a third-order force that arises from computing the drag loads considering the instantaneous wave elevation (as opposed to the approach considered in this work, which integrates drag loads up to the mean water level) explained the underprediction of the slow surge motion reported by \cite{OC52017}. 

\begin{figure}[!hbtp]
	\centering
	\includegraphics[width=\textwidth]{./figures/ptfmsurge-drag_pontoon.png}	
	\caption{Illustration of the influence on surge motions of the drag forces on the pontoons (IRR-I1 wave, incidence of 60\textdegree{}, without wind effects). The bottom right plot is a zoom in linear scale of the PSD around the resonance frequency of surge (about $0.006\,\text{Hz}$). The range of frequencies corresponding to the incoming waves is shaded in blue.} \label{fig:exp_vs_num:drag:ptfmsurge}
\end{figure}

Though the effect on surge was important, the impact on heave is more critical, as illustrated in Figure~\ref{fig:exp_vs_num:drag:ptfmheave}. In this case, drag acts not only by damping the resonant response around $16.5\,\text{s}$ ($0.061\,\text{Hz}$), but also as a relevant forcing term around $15\,\text{s}$ ($0.067\,\text{Hz}$). This is made clear by the Heave RAOs\footnote{Obtained from the experiments and OpenFAST, which are in time domain, for the IRR-I1 wave.} plotted on the bottom right of the same figure, which show that the response predicted by WAMIT is almost zero at $15\,\text{s}$, a consequence of the force obtained with potential flow being null close to that period, while the result with OpenFAST considering the rectangular pontoon adheres very well to the experiment. This means that it would not be possible to model this behavior by tuning an external quadratic damping coefficient, as neglecting this forcing term would lead to important underpredictions of the motions. 

Still about heave, it is noteworthy that a single set of drag coefficients, obtained from such a simple procedure as a decay test and using an equally simple approach as Morison's equation, was able to model pretty well both the situation in which drag acts by damping the motion and the one in which it plays the role of a forcing term.
\begin{figure}[!hbtp]
	\centering
	\includegraphics[width=\textwidth]{./figures/ptfmheave-drag_pontoon.png}	
	\caption{Illustration of the influence on heave motions of the drag forces on the pontoons (IRR-I1 wave, incidence of 60\textdegree{}, without wind effects). The range of frequencies corresponding to the incoming waves is shaded in blue.} \label{fig:exp_vs_num:drag:ptfmheave}
\end{figure}








\subsection{The importance of second-order forces on both horizontal and vertical motions} \label{subsec:exp_vs_num:2nd_order}
Mostrar o offset e o pitch

\subsection{The impact of mean hull inclination when computing  radiation/diffraction coefficients} \label{subsec:exp_vs_num:impact_inclination}
As mentioned in Section~\ref{sec:numerical_models}, one of the objectives of this work is to assess the impact of considering the mean hull inclination caused by the wind when solving the radiation/diffraction problem. \textcolor{red}{Figure~X}, which summarizes in a boxplot the differences in the maxima obtained for each of the quantities analyzed in the previous sections for all the irregular waves, shows that this is not the case: in fact, the differences are .......... (falar também que a diferença é ainda mais irrelevante quando se pensa na tabela de extremos)

**Mostrar um gráfico comparando as estatísticas calculadas c/ inclinação e sem.**

As a more in-depth example, \textcolor{red}{Figure~X} presents the time series and PSD's of roll and pitch motion obtained for the FOWT under the combined action of the IRR12 sea ($H_S=4.44\,\text{m}$, $T_P = 11.34\,\text{s}$ and incidence of -10\textdegree{}) and the turbulent wind condition (mean wind speed $10.59\,\text{s}$ and $\textrm{TI}=12\%$) with an incidence of 47\textdegree{}, which is schematized in the same figure. This case was chosen for being the one that presented the largest difference in the horiontal acceleration at the nacelle, with the model considering an inclined mesh (denoted by IC) predicting a maximum horizontal acceleration of $0.85\,\text{m}/\text{s}^2$ and the one with an even keel mesh (denoted by EK) providing $0.74\,\text{m}/\text{s}^2$, which is actually closer to the experimental value of $0.64\,\text{m}/\text{s}^2$. 

**Mostrar gráfico de série temporal do que deu a maior diferença e explicar usando RAO e o .3**

In fact, this could be anticipated by looking directly at the radiation/diffraction coefficients that are imported by OpenFAST. These are illustrated by \textcolor{red}{Figure~X} (first-order diffraction forces), \textcolor{red}{Figure~X} (mean drift force) and \textcolor{red}{Figure~X} (added mass and potential damping). For conciseness, only the mesh with the largest inclination (\textcolor{red}{dizer qual é aqui, i.e. p/ qual vento, e qual é a inclinação}) and only one wave incidence (45\textdegree{}) is plotted, and the results for sway and roll are omitted because they are qualitatively the same as surge and pitch.

**Mostrar gráficos das forças**

It is worth pointing out that only the impact on the radiation/diffraction coefficients was assessed, and it is possible that hull inclination may be important to effects related to real flow phenomena, such as drag forces.


\subsection{Main results} \label{subsec:exp_vs_num:main_results}
Explicar o procedimento adotado para processar a grande quantidade de ondas, que é baseado nas estatísticas.
Mostrar tabela com períodos naturais e níveis de amortecimentos + tabela de resumo das estatísticas p/ ondas irregulares.

Ilustrar c/ gráficos de séries temporais e espectros de casos selecionados (tem que ter a aerodinâmica p/ mostrar que o rotor tá funcionando) + gráfico do máximo e média



\section{The impact of model simplifications on the response of the FOWT} \label{sec:impact_simplifications}
- Comparar resultados das simulações nas condições reais e identificar diferenças pro modelo que é mais próximo do ensaio.

- Usar simulações intermediárias p/ explicar essas diferenças

%\subsection{The impact of considering only aerodynamic thrust}
%
%\subsection{The impact of neglecting blade and tower elasticity}

\input{sections/conclusions}

% To print the credit authorship contribution details
\printcredits

\section*{Declaration of competing interest}
The authors declare that they have no known competing financial interests or personal relationships that could have appeared to influence the work reported in this paper.

\section*{Acknowledgments}
This study was financed in part by the Coordenação de Aperfeiçoamento de Pessoal de Nível Superior - Brasil (CAPES) - Finance Code 001. Alexandre Simos thanks the Brazilian National Council for Scientific and Technological Development - CNPq - for his research grant (\# 306342/2020-0). 

%% Loading bibliography style file
%\bibliographystyle{model1-num-names}
\bibliographystyle{cas-model2-names}

\bibliography{mybibfile}


%\appendix
%\input{sections/appendices}

\end{document}

