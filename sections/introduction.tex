\section{Introduction} \label{sec:introduction}
Ensaio com SIL com algumas limitações

Na primeira parte, a ideia é verificar aspectos importantes da hidrodinâmica (que é a parte que é fisicamente capturada no ensaio) que o modelo numérico tem que levar em conta. Mais especificamente, é avaliada a importância de elementos de Morison retangulares p/ modelagem do pontoon; a importância de forças de segunda-ordem tanto na horizontal quanto vertical (conforme já sabido na literatura); e mostra-se que levar em conta a inclinação média do casco devido ao vento não é mto importante.

Na segunda parte, o objetivo é avaliar o quão importante são aspectos que foram deixados de lado na modelagem aerodinâmica do ensaio, o que é feito numericamente comparando o modelo que é fiel às condições de ensaio (apenas thrust e pás consideradas rígidas) com um modelo numérico em que as forças aerodinâmicas são calculadas nos seis graus de liberdade e a flexibilidade das pás é considerada (embora de forma simples com o elastodyn. Preciso estudar em que situações seria necessário usar o beamdyn).