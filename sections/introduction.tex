\section{Introduction} \label{sec:introduction}
Floating offshore wind turbines (FOWTs) have been the subject of numerous studies due to the possibility of exploiting the vast wind resources located in deep waters. As an emerging technology, the growth of the wind energy industry depends on FOWTs achieving more competitive costs, which has pushed for larger rotors and new designs for both floaters and moorings.

Since FOWTs are complex structures, their design requires the evaluation of performance and structural integrity for a myriad of environmental conditions (wind, wave, current, among others) and operating conditions (power production, normal shut down, fault conditions, etc.). Due to their intricate dynamics, this procedure requires modelling software capable of accounting for the couplings between aerodynamics, hydrodynamics, controls, moorings and structural behavior, which are commonly referred as aero-hydro-servo-elastic tools. A substantial effort has been made to validate these software, as exemplified by the OC3~\citep{jonkman2010report}, OC4~\citep{OC42014} and OC5~\citep{OC52017} projects, but this is still an ongoing development.

In fact, the experiments required to validate the numerical tools, usually performed in model scale, are far from an easy task, for it is impossible to keep all the dimensionless parameters that describe the different physical aspects of the problem. For instance, while the scaling of the waves requires that the Froude number ($\textrm{Fr} = U^2/(gL)$, with $U$ a characteristic speed, $L$ a characteristic length and $g$ the gravitational acceleration) be conserved, the aerodynamic loads are governed by the Reynolds number ($\textrm{Re} = UL/\nu$, with $\nu$ the kinematic viscosity). To work around this incompatibility, some alternatives have been tried to perform tests with both wind and waves, and a thorough review of experimental techniques for floating offshore wind turbines can be found in \textcolor{red}{otter}. For instance, some works have used a Froude scaled rotor with higher wind speeds, with the wind generated by fans, in order to obtain the correct rotor thrust \textcolor{red}{111, 115, 116}, but this approach has the downside that either the tip speed ratio (TSR) or the excitation frequencies are not preserved. Others have employed performance scaled rotors \textcolor{red}{3, 120, 121, 122}, in the sense that the rotors were redesigned with geometrically modified airfoils to compensate for the low Reynolds number obtained in a Froude scale experiment. 

A different line of thought 



O SIL com atuador é muito simples em questão de equipamento, já que não precisa de ventiladores e nem das winches


- ROTORES EM ESCALA, ETC. 
--- Roddier et al, 2010 -> Drag disc
--- Martin et al, 2014 
    -> Froude scaled rotor with higher wind speeds, but wrong thrust
    -> Increase rotor velocity to keep TSR, changes excitation frequencies
    -> Geommetrically modified airfoils to compensate for low Reynolds Number -> Geommetrically scaling of the rotor

- SIL C/ vento

- E OS COM CABOS


Vittori et al, 2022: SIL c/ thruster
Thys et al, 2021: Cabos



Explicar nosso objetivo, que é duplo: verificar nossos modelos numéricos e, concomitantemente, desenvolver a capacidade do tanque de provas numérico em realizar esse tipo de ensaio.

Ensaio com SIL com algumas limitações

Na primeira parte, a ideia é verificar aspectos importantes da hidrodinâmica (que é a parte que é fisicamente capturada no ensaio) que o modelo numérico tem que levar em conta. Mais especificamente, é avaliada a importância de elementos de Morison retangulares p/ modelagem do pontoon; a importância de forças de segunda-ordem tanto na horizontal quanto vertical (conforme já sabido na literatura); e mostra-se que levar em conta a inclinação média do casco devido ao vento não é mto importante.

Na segunda parte, o objetivo é avaliar o quão importante são aspectos que foram deixados de lado na modelagem aerodinâmica do ensaio, o que é feito numericamente comparando o modelo que é fiel às condições de ensaio (apenas thrust e pás consideradas rígidas) com um modelo numérico em que as forças aerodinâmicas são calculadas nos seis graus de liberdade e a flexibilidade das pás é considerada (embora de forma simples com o elastodyn. Preciso estudar em que situações seria necessário usar o beamdyn).