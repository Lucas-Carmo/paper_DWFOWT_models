\section{Reproducing the experiments with numerical models} \label{sec:exp_vs_num}

\subsection{The need for drag forces on the pontoons}
Mostrar os decaimentos com três curvas: experimental, $C_D$ pro caso de um pontoon circular (o errado, tipo heave only p/ surge e surge only p/ heave) e $C_D$ pro pontoon retangular. Daí, mostrar que consegue pegar bem o heave e o surge simultaneamente quando tá retangular, o que não é possível no caso circular.

\subsection{The importance of second-order forces on both horizontal and vertical motions}
Mostrar o offset e o pitch

\subsection{The impact of mean hull inclination when computing radiation/diffraction coefficients}
Comparar as forças calculadas no WAMIT (1a e 2a ordem) p/ o caso de maior inclinação e o de menor.

Mostrar alguns RAOs selecionados.

Mostrar um gráfico comparando as estatísticas calculadas c/ inclinação e sem.



\subsection{Response under the action of irregular waves and wind}
Explicar o procedimento adotado para processar a grande quantidade de ondas, que é baseado nas estatísticas.

Mostrar a tabela com as estatísticas.

Ilustrar c/ gráficos de séries temporais e espectros de casos selecionados (tem que ter a aerodinâmica p/ mostrar que o rotor tá funcionando) + gráfico do máximo e média