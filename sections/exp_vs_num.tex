\section{Reproducing the experiments with numerical models} \label{sec:exp_vs_num}
Explicar aqui como é organizada a seção e a metodologia usada p/ mostrar o resultado. O foco é em apresentar três aspectos físicos que tiveram maior atenção na modelagem do problema, e não tanto nos resultados da comparação entre o experimento e o ensaio.


\subsection{The need for drag forces on the pontoons}
Fazer figuras ilustrativas p/ mostrar o pontoon e a nomenclatura, junto com uma tabela com os diâmetros e coeficientes de arrasto adotados nas colunas/pontoons p/ cada um.

Mostrar os decaimentos com três curvas: experimental, $C_D$ pro caso de um pontoon circular (o errado, tipo heave only p/ surge e surge only p/ heave) e $C_D$ pro pontoon retangular. Daí, mostrar que consegue pegar bem o heave e o surge simultaneamente quando tá retangular, o que não é possível no caso circular.


Resultados a gerar:
- Gráfico do Decaimento de heave com (lado a lado com o de surge):
--- Experimental
--- OpenFAST - Rect. pontoon
--- OpenFAST - Circ. pontoon S 
- Gráfico do Decaimento de surge com (lado a lado com o de heave):
--- Experimental
--- OpenFAST - Rect. pontoon
--- OpenFAST - Circ. pontoon H 

- Heave p/ APR01-IDLE-IRR-I1
-- Série temporal e espectro na esquerda
-- RAO na direita
--- Experimental
--- OpenFAST - Rect. pontoon
--- OpenFAST - Circ. pontoon S 
--- WAMIT no RAO

- Mesma coisa p/ surge


\subsection{The importance of second-order forces on both horizontal and vertical motions}
Mostrar o offset e o pitch

\subsection{Main results}
Explicar o procedimento adotado para processar a grande quantidade de ondas, que é baseado nas estatísticas.
Mostrar tabela com períodos naturais e níveis de amortecimentos + tabela de resumo das estatísticas p/ ondas irregulares.

Ilustrar c/ gráficos de séries temporais e espectros de casos selecionados (tem que ter a aerodinâmica p/ mostrar que o rotor tá funcionando) + gráfico do máximo e média


\subsection{The impact of mean hull inclination when computing radiation/diffraction coefficients} \label{subsec:impact_inclination}
As mentioned in Section~\ref{sec:numerical_models}, one of the objectives of this work is to assess the impact of considering the mean hull inclination caused by the wind when solving the radiation/diffraction problem. \textcolor{red}{Figure~X}, which summarizes in a boxplot the differences in the maxima obtained for each of the quantities analyzed in the previous sections for all the irregular waves, shows that this is not the case: in fact, the differences are .......... (falar também que a diferença é ainda mais irrelevante quando se pensa na tabela de extremos)

**Mostrar um gráfico comparando as estatísticas calculadas c/ inclinação e sem.**

As a more in-depth example, \textcolor{red}{Figure~X} presents the time series and PSD's of roll and pitch motion obtained for the FOWT under the combined action of the IRR12 sea ($H_S=4.44\,\text{m}$, $T_P = 11.34\,\text{s}$ and incidence of -10\textdegree{}) and the turbulent wind condition (mean wind speed $10.59\,\text{s}$ and $\textrm{TI}=12\%$) with an incidence of 47\textdegree{}, which is schematized in the same figure. This case was chosen for being the one that presented the largest difference in the horiontal acceleration at the nacelle, with the model considering an inclined mesh (denoted by IC) predicting a maximum horizontal acceleration of $0.85\,\text{m}/\text{s}^2$ and the one with an even keel mesh (denoted by EK) providing $0.74\,\text{m}/\text{s}^2$, which is actually closer to the experimental value of $0.64\,\text{m}/\text{s}^2$. 

**Mostrar gráfico de série temporal do que deu a maior diferença e explicar usando RAO e o .3**

In fact, this could be anticipated by looking directly at the radiation/diffraction coefficients that are imported by OpenFAST. These are illustrated by \textcolor{red}{Figure~X} (first-order diffraction forces), \textcolor{red}{Figure~X} (mean drift force) and \textcolor{red}{Figure~X} (added mass and potential damping). For conciseness, only the mesh with the largest inclination (\textcolor{red}{dizer qual é aqui, i.e. p/ qual vento, e qual é a inclinação}) and only one wave incidence (45\textdegree{}) is plotted, and the results for sway and roll are omitted because they are qualitatively the same as surge and pitch.

**Mostrar gráficos das forças**

It is worth pointing out that only the impact on the radiation/diffraction coefficients was assessed, and it is possible that hull inclination may be important to effects related to real flow phenomena, such as drag forces.