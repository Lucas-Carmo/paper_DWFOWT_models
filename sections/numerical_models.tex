\section{Numerical models} \label{sec:numerical_models}
CODIGO MODIFICADO P/ TER PONTOONS

CODIGO MODIFICADO P/ FORCAS AERODINAMICAS APENAS NO EIXO X



The wave forces are considered in OpenFAST by a combination of radiation/diffraction forces, taken into account using Cummins' approach \citep{cummins1962,ogilvie1964} with frequency-domain coefficients computed with WAMIT (\textcolor{red}{version 7.0.1}), and the quadratic drag from Morison's equation. Concerning the former, one of the main questions about the numerical modeling of the experiments was whether the mean hull inclination caused by the wind should be considered when solving the radiation/diffraction problem. Indeed, one of the main hypothesis of the Boundary Element Method behind WAMIT is that the body oscillates around a mean position, but it is not clear at first how important the few degrees of inclination induced by the wind are.

For that reason, a different set of radiation/diffraction coefficients (i.e. first- and second-order wave forces, added mass and potential damping) was computed for each wind condition, using low order meshes with different inclinations that were determined by experimentally measuring the inclination of the model under the action of constant wind in calm waters. Since this is a somewhat cumbersome procedure, it is important to assess whether it is worth the cost, so all the OpenFAST simulations were performed twice: once with radiation/diffraction coefficients obtained using the inclined mesh, and once with coefficients from an even keel mesh. One of the inclined meshes (\textcolor{red}{ESPECIFICAR}) and the even keel mesh are illustrated in \textcolor{red}{Figures~XX}, while the differences between the results obtained with them are discussed in Section~\ref{subsec:exp_vs_num:impact_inclination}.
